\documentclass[12pt, twoside, hidelinks, a4paper]{article}


\usepackage[]{geometry}
\geometry{inner=30mm, outer=20mm, top=25mm, bottom=25mm}

\usepackage{mystyle}
\pagestyle{headings}

\usepackage{fancyhdr}
\fancyhf{}
\pagestyle{fancy}
\renewcommand{\headrulewidth}{0pt}
% numery stron: lewa do lewego, prawa do prawego
\fancyfoot[LE,RO]{\thepage}

\fancypagestyle{plain}
{
   \fancyhf{}
\renewcommand{\headrulewidth}{0pt}
% numery stron: lewa do lewego, prawa do prawego
\fancyfoot[LE,RO]{\thepage}
}

\usepackage{pdfpages}

%\renewcommand{\familydefault}{\sfdefault}
\setlength\parindent{1cm}

\usepackage{indentfirst}
\usepackage[affil-it]{authblk}

\begin{document}
    \setstretch{1.15}
 	\pagenumbering{arabic}
    %\include{chapters/abstract}

\author{Joanna Czechura, Tomasz Kowalczyk, Marcin Waszak}
\title{Bezprzewodowe sieci sensorów – systemy wbudowane czasu rzeczywistego}
\date{\today}
\affil{Wydział Elektroniki i Technik Informacyjnych, Politechnika Warszawska}


\maketitle
    
    %\include{chapters/introduction}

\begin{abstract}
Lorem ipsum dolor sit amet, consectetur adipiscing elit, sed do eiusmod tempor incididunt ut labore et dolore magna aliqua. Ut enim ad minim veniam, quis nostrud exercitation ullamco laboris nisi ut aliquip ex ea commodo consequat. 
\end{abstract}
\section{First bit}
If you want to ramp your text straight onto the title page, start the text at 
something that does not cause a page break, like a section.  Here's a handy 
place to introduce some of your woofy conventions, like quotes in equations.
\subsection{New Page}
A new chapter starts a new page. \cite{przepior}

    %\pagenumbering{gobble}
    %\setcounter{page}{2}
    \printbibliography

\end{document}
